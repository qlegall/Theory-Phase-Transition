%Report template

\documentclass[10pt,a4paper]{amsart}

% Class options



\usepackage[utf8]{inputenc}
\usepackage{enumitem}
%\usepackage[latin1]{inputenc}
\usepackage[T1]{fontenc}
\usepackage{graphicx}
\usepackage{latexsym}
\usepackage{amsmath,amssymb, amsfonts, amsthm, stmaryrd, enumitem, mathtools}
\usepackage{ifpdf}
%\usepackage{natbib}
%\usepackage{mathabx} % bigtimes
\usepackage{mathrsfs} % mathscr
\usepackage{subfig}
\usepackage{float}
\usepackage{color}
\usepackage{hyperref}
% plotting
\usepackage{pgfplots}
\pgfplotsset{compat=newest}
\pgfplotsset{every axis legend/.append style={
at={(0,0)},
anchor=north east}} 


% abbreviations
%\usepackage[intoc]{nomencl}
%\renewcommand{\nomname}{Notations}



% Modify top margin 
%\setlength{\topmargin}{-15mm}

% Theorems, definitions...
\usepackage{amsthm}
\usepackage{bbm}

\newtheoremstyle{exampstyle}
{15pt} % Space above
{15pt} % Space below
{} % Body font. Remove itshape if want texts of theorems normal and not in italics
{} % Indent amount
{\bfseries} % Theorem head font
{.} % Punctuation after theorem head
{.5em} % Space after theorem head
{} % Theorem head spec (can be left empty, meaning `normal')

\theoremstyle{exampstyle}

\newtheorem{Theorem}{Theorem}
\newtheorem{Definition}{Definition}
\newtheorem{Lemma}{Lemma}
\newtheorem{Remark}{Remark}
\newtheorem{Example}{Example}
\newtheorem{Corollary}{Corollary}
\newtheorem{Proposition}{Proposition}
%\newtheorem{Notation}{Notation}
\newenvironment{Proof}{\noindent{\textit{Proof.}}}{%
\hspace*{\fill}$\Box$\par\vskip2ex}
\newcommand{\bspend}{%
\hspace*{\fill}$\blacktriangleleft$\par\vskip2ex}
\newcommand{\bewend}{%
\hspace*{\fill}$\Box$\par\vskip2ex}

%\newtheorem{Theorem}[Theorem]{Theorem}
%\newtheorem{Definition}[Theorem]{Definition}
%\newtheorem{Lemma}[Theorem]{Lemma}
%\newtheorem{Remark}[Theorem]{Remark}
%\newtheorem{Example}[Theorem]{Example}
%\newtheorem{Corollary}[Theorem]{Corollary}
%\newtheorem{Proposition}[Theorem]{Proposition}
%\newtheorem{Notation}{Notation}
%\newenvironment{Proof}{\noindent{\textit{Proof.}}}{%
%\hspace*{\fill}$\Box$\par\vskip2ex}
%\newcommand{\bspend}{%
%\hspace*{\fill}$\blacktriangleleft$\par\vskip2ex}
%\newcommand{\bewend}{%
%\hspace*{\fill}$\Box$\par\vskip2ex}

\newtheoremstyle{exampnotations}
{15pt} % Space above
{15pt} % Space below
{} % Body font. Remove itshape if want texts of theorems normal and not in italics
{} % Indent amount
{\bfseries} % Theorem head font
{.} % Punctuation after theorem head
{.5em} % Space after theorem head
{} % Theorem head spec (can be left empty, meaning `normal')

\theoremstyle{exampnotations}  
\newtheorem{Notation}{Notation}

\usepackage{caption}

\usepackage{newclude}

\usepackage[final]{pdfpages}

\usepackage{fancyhdr}
\fancyhead{}
\fancyhead[RO]{\slshape \rightmark}
%\fancyhead[LE]{\slshape \leftmark}

%\allowdisplaybreaks
% instead: single \displaybreak in align-environment

\usepackage{hyperref}
%\hypersetup{
% colorlinks,
% citecolor=black,
% filecolor=black,
% linkcolor=black,
% urlcolor=black
%}

% macros
\newcommand{\rpos}{\mathbb{R}_\geq}
\newcommand{\Z}{\mathbb{Z}}
\newcommand{\R}{\mathbb{R}}
\newcommand{\N}{\mathbb{N}}
\newcommand{\E}{\mathbb{E}}
\newcommand{\kn}{\kappa_n}
\newcommand{\pl}{\left( \left\{ }
\newcommand{\pr}{ \right\} \right) }
\newcommand{\supt}{\sup_{0\leq t \leq T}}
\newcommand{\il}{\left( } % integrals
\newcommand{\ir}{\right) }
\newcommand{\Il}{\Bigg(} % integrals
\newcommand{\Ir}{\Bigg)}
\newcommand{\el}{\left[ }
\newcommand{\er}{\right] }
\newcommand{\El}{\Bigg[ }
\newcommand{\Er}{\Bigg] }
\newcommand{\rl}{\left(}
\newcommand{\rr}{\right)}
\newcommand{\Rl}{\Bigg(}
\newcommand{\Rr}{\Bigg)}
\newcommand{\vl}{\left|}
\newcommand{\vr}{\right|}
\newcommand\floor[1]{\lfloor#1\rfloor}
\newcommand\ceil[1]{\lceil#1\rceil}
\newcommand{\indep}{\rotatebox[origin=c]{90}{$\models$}}


%\makenomenclature

% Hyphenation
\hyphenation{Lip-schitz}
\hyphenation{sto-chas-tic}
\hyphenation{nu-me-ri-cal}
\hyphenation{appro-xi-ma-tion}
\hyphenation{accor-ding}
\hyphenation{mathematics}
\hyphenation{in-equa-li-ty}
\hyphenation{in-equa-li-ties}
\hyphenation{mea-sura-bi-li-ty}
\hyphenation{Brownian}
\hyphenation{Example}
\hyphenation{si-mu-la-tion}
\hyphenation{si-mu-la-tions}

% Appendices
%\usepackage[toc,page]{appendix}



% Colours
\definecolor{dark1}{RGB}{27, 158, 119}
\definecolor{dark2}{RGB}{217, 95, 2}
\definecolor{dark3}{RGB}{117, 112, 179}
\definecolor{dark4}{RGB}{231, 41, 138}  

%\usepackage{chngcntr}
%\counterwithout{footnote}{chapter}

\newcommand{\vertiii}[1]{{\left\vert\kern-0.25ex\left\vert\kern-0.25ex\left\vert #1 \right\vert\kern-0.25ex\right\vert\kern-0.25ex\right\vert}}

%\usepackage{tocloft}
%\setlength\cftbeforefigskip{\cftbeforechapskip}
%\setlength\cftbeforetabskip{\cftbeforechapskip}

\begin{document}
\title{Percolation for Cox Point Processes with Canyon Shadowing}
\author[Q. Le Gall]{Quentin Le Gall}
\address{Orange Labs Networks, 44 avenue de la République 92320 Ch\^atillon}
\email{quentin1.legall@orange.com}
\author[B. B\l{}aszczyszyn] {Bart\l{}omiej B\l{}aszczyszyn}
\address{Inria - \MakeUppercase{ens}, 2 rue Simone Iff CS42112 75589 Paris Cedex 12}
\email{bartek.blaszczyszyn@ens.fr }
\author[E. Cali]{\'Elie Cali}
\address{Orange Labs Networks, 44 avenue de la République 92320 Ch\^atillon}
\email{elie.cali@orange.com}
\author[T. En-Najjary]{Taoufik En-Najjary}
\address{Orange Labs Networks, 44 avenue de la République 92320 Ch\^atillon}
\email{taoufik.ennajjary@orange.com}
\maketitle
\tableofcontents
\section{Foreword}
\subsection{Network model}
\label{Ss.NetworkModel}
Consider a probability space $(\Omega, \mathcal{F}, \mathbb{P})$ and state space $(\mathbb{R}^{2}, \mathcal{B}(\mathbb{R}^{2}))$, where $\mathcal{B}(\mathbb{R}^{2})$ is the usual Borel $\sigma$-algebra of $\mathbb{R}^{2}$. \\

Let $\lambda_{S} > 0$ and $X_{S}$ be a homogeneous planar Poisson point process (PPP) in the state space $\mathbb{R}^{2}$ with intensity $\lambda_S$. Consider the Poisson-Voronoi tessellation (PVT) $S$ associated with $X_S$. In particular, $S$ is stationary. By analogy with a telecommunications network, $S$ will be called \emph{street system} from now owards. \\
\indent Denote by $E \coloneqq (e_{i})_{i \geq 1}$ the edge-set of $S$ and by $V \coloneqq (V_{i})_{i \geq 1}$ the vertex-set of $V$. Furthering the aforementioned analogy, the elements of $E$ (resp. $V$) will be called \emph{street segments} (resp. \emph{crossroads}). \\
\indent Let $\Lambda$ be the stationary random measure defined such that:
\begin{itemize}
\item $\mathbb{E} \Lambda \left[0,1\right]^{2} = 1$
\item $\Lambda(dx) = \nu_{1}(S \cap dx)$, where $\nu_{1}$ denotes the 1-dimensional Hausdorff measure of $\mathbb{R}^{2}$. In other words, $\nu_{1}(S \cap B)$ is the total edge length of $S$ contained in the Borel observation window $B$ and so $\Lambda$ can be seen as a Lebesgue-like measure on the edges of $S$, rescaled in such a way that the total $\Lambda$-measure of a 1-area window is 1. \\
\end{itemize}

The key network parameters are:
\begin{itemize}
\item The user density $\lambda > 0$.
\item The relay proportion $p \in \left(0,1\right)$.
\item The connectivity radius $r > 0$. \\
\end{itemize}

The users, equipped with mobile devices, are modelled by a Cox process $X^{\lambda}$ driven by the random intensity measure $\lambda \Lambda$. In other words, conditioned on a given realization of the street system $S$, $X^{\lambda}$ is a PPP with mean measure $\lambda \Lambda$. In particular, the number of users on a given street segment $e \in E$ is a Poisson random variable with mean $\nu_{1}(e)$ and the numbers of users in two disjoint subsets of $E$ are independent random variables. \\
\indent The relays (either representing physical antennas or additional users not modelled through $X^{\lambda}$) are modelled by a doubly stochastic Bernoulli point process $Y$ on the set of crossroads $V$ with parameter $p$, so that one can write:
\begin{equation*}
Y = \sum_{i} \mathbbm{1}_{\lbrace U_{i} \leq p \rbrace } \delta_{V_{i}},
\end{equation*}
where $U_{i} \;\substack{ \text{i.i.d.} \\ \sim} \; \mathcal{U}(0,1)$ and where $\delta_{x}$ denotes the Dirac measure at $x$. In other words, conditioned on $\Lambda$ (or, equivalently, $S$) each crossroad is retained (resp. erased) independently from all others with probability $p$ (resp. $1-p$). \\
\noindent Moreover, we also assume that the processes of users and of relays  are conditionnally independent given their random support, i.e. $X^{\lambda} \indep Y \, \vert \,  \Lambda$. We denote $Z \coloneqq X^{\lambda} \cup Y$ the superposition of users and relays. 
%Then, conditioned on $\Lambda$, $Z$ has an independence property similar to the one of Poisson processes:
%\begin{Proposition}
%Given $\Lambda$, $Z$ has the complete independence property.
%\end{Proposition}
%\begin{Proof}
%Let $A, B \subset \mathbb{R}^{2}$ be Borel sets such that $A \cap B = \emptyset$. We need to show that $Z(A) \indep Z(B) \, \vert \, \Lambda$.
%\end{Proof}
%\vspace{2\baselineskip}
The network is modelled by the \emph{connectivity graph} $\mathcal{G}_{r,\lambda,p}$ defined in the following way:
\begin{itemize}
\item $\mathcal{G}_{r,\lambda,p}$ is undirected.
\item The vertex set of $\mathcal{G}_{r,\lambda,p}$ are the points of $Z$.
\item The edge $Z_{i} \leftrightsquigarrow Z_{j}, i \neq j$ is drawn if and only if $Z_{i}$ and $Z_{j}$ are located on the \emph{same} street segment and of mutual Euclidean distance less than $r$. In other words:
\begin{equation}
\label{connectivity_mechanism}
\forall \, i \neq j, \: Z_{i} \leftrightsquigarrow Z_{j} \Leftrightarrow 
\left\{
\begin{array}{l}
\exists \, e \in E, \, Z_{i} \in e  \  \text{and} \  Z_{j} \in e \\
\lVert Z_{i} - Z_{j} \rVert \leq r
\end{array}
\right.
\end{equation}
\end{itemize}

%\begin{Remark}
%The connectivity mechanism defined by \eqref{connectivity_mechanism} only allows for so-called line-of-sight (LOS) connections, which can be seen as a modelling of physical obstacles preventing connections to occur in all directions.
%\end{Remark}

\paragraph*{Main question :} Percolation regime of the connectivity graph $\mathcal{G}_{r,\lambda, p}$? Are there critical values of the parameters $(r,\lambda,p)$ for which percolation of $\mathcal{G}_{r,\lambda, p}$ occurs with positive probability?

\subsection{Definitions and Notations}
We begin with introducing a few notations and definitions useful for the purposes of our developments.
 \begin{Notation}
 For $A \subset \mathbb{R}^{2}$ and $B \subset \mathbb{R}^{2}$, we denote as customary the Euclidean distance between $A$ and $B$ by:
 $$\text{dist}(A,B) \coloneqq \inf \lbrace \lVert x - y \rVert_{2} , \,  x \in A, \,  y \in B \rbrace$$
 \end{Notation}

\begin{Notation}
For $x \in \mathbb{R}^{2}$, $n \in \mathbb{N} \setminus \lbrace 0 \rbrace$, we denote by $$Q_n(x) \coloneqq x + \left[-n/2,n/2\right]^{2}$$ the square of side $n$ centered at $x$. We note that this is exactly the definition of the closed ball with center $x$ and radius $n/2$ for the infinite norm of $\mathbb{R}$
\end{Notation}

\begin{Notation}
For $n \in \mathbb{N} \setminus \lbrace 0 \rbrace$, we will write 
$Q_n$ to mean $Q_n(0)$.
\end{Notation}

\begin{Notation}
We denote by $\mathbf{M}$ the space of Borel measures on $\mathbb{R}^{2}$, equipped with the evalutation $\sigma$-algebra \cite[Section 13.1]{last2017lectures}, i.e. the smallest $\sigma$-algebra making the mappings $\Xi \mapsto \Xi(B)$ measurable for all Borel sets $B \subset \mathbb{R}^{d}$.
\end{Notation}

\begin{Notation}
For a (possibly random) Borel measure $\mu$ on $\mathbb{R}^{2}$ and $A \subset \mathbb{R}^{2}$, we denote the restriction of $\mu$ to $A$ by: $$\mu_A(\cdot) \coloneqq \mu(A \cap \cdot)$$
\end{Notation}

Finally, we will use the following convenient notation for the length of a street segment or a subset of street segment:
\begin{Notation}
Let $e \in E$ and $s \subseteq e$. Then we denote the length of $s$ by $\vert s \vert$.
\end{Notation}

\begin{Definition}
Let $\mu$ be a (possibly random) Borel measure on $\mathbb{R}^{2}$. The \emph{support} of $\mu$ is the following set:
$$\text{supp}(\mu) \coloneqq \lbrace x \in \mathbb{R}^{d} \, : \forall \varepsilon > 0, \,  \mu(Q_{\varepsilon}(x)) > 0 \rbrace$$
\end{Definition}
We will then need the concepts of \emph{stabilization} and \emph{essential asymptotical connectedness} given in \cite{hirsch_continuum_2017} for investigating spatial dependencies of random measures:

\begin{Definition}\cite[Definition 2.3]{hirsch_continuum_2017}
\label{Def.stabilizing}
A random measure $\Xi$ on $\mathbb{R}^{2}$ is called \emph{stabilizing} if there exists a random field of stabilisation radii $R = \lbrace R_{x} \rbrace _{x \in \mathbb{R}^{2}}$ defined on the same probability space as $\Xi$ such that:
\begin{enumerate}[label = (\arabic*)]
\item $(\Xi,R)$ are jointly stationary
\item $\displaystyle \lim_{n \uparrow \infty} \mathbb{P}\left(\sup _{y \in Q_{n} \cap \mathbb{Q}^{2}} R_{y} < n\right) = 1$
\item for all $n \geq 1$, the random variables 
$$ \left\lbrace f\left(\Xi_{Q_{n}(x)}\right)\mathbbm{1}\left\lbrace \sup_{y \in Q_{n}(x) \cap \mathbb{Q}^{2}} R_{y} < n \right\rbrace \right\rbrace _{x \in \varphi}$$
are independent for all bounded measurable functions $$f : \mathbf{M} \to \left[0, +\infty \right)$$ and finite $\varphi \subset \mathbb{R}^{2}$ such that $\forall \, x \in \varphi, \, \text{dist}(x, \varphi \setminus \lbrace x \rbrace ) > 3n$.
\end{enumerate}
\end{Definition}
\begin{Remark}
Throughout the rest of this paper, we assume the random variables $(R_x)_{x \in \mathbb{R}^{2}}$ to be $\Lambda$-measurable, as has been done by the authors of \cite{hirsch_continuum_2017}.
\end{Remark}
\begin{Remark}
In the above definition, the random field of stabilization radii $R = \lbrace R_{x} \rbrace _{x \in \mathbb{R}^{2}}$ may not be unique, i.e. a random measure can be stabilizing for more than one field of stabilization radii.
\end{Remark}
\begin{Remark}
Let $b > 0$ and assume that $\Xi$ is $b$-dependent, i.e. $$\forall A,B \subset \mathbb{R}^{d}, \, \text{dist}(A,B) > b \Rightarrow \Xi_{A} \indep \Xi_{B}$$ Then $\Xi$ is stabilizing. 
\end{Remark} 

\begin{Definition}\cite[Definition 2.5]{hirsch_continuum_2017}
Let $\Xi$ be a random measure. Then $\Xi$ is \emph{essentially asymptotically connected} if there exists a random field $R = \lbrace R_{x} \rbrace_{x \in \mathbb{R}^{d}}$ such that $\Lambda$ is stabilizing for $R$ and if for all $n \geq 1$, whenever $\displaystyle \sup_{y \in Q_{2n} \cap \mathbb{Q}^{2}} R_{y} < n/2$, the following assertions are satisfied:
\begin{enumerate}[label = (\arabic*)]
\item $\text{supp}(\Xi_{Q_{n}}) \neq \emptyset$
\item $\text{supp}(\Xi_{Q_{n}})$ is contained in a connected component of $\text{supp}(\Xi_{Q_{2n}})$
\end{enumerate}
\end{Definition}

The following result from \cite{hirsch_continuum_2017} will turn out to be useful for our purposes:

\begin{Proposition}\cite[Example 3.1]{hirsch_continuum_2017}
Let $\Lambda = \nu_{1}(S \cap dx)$, where $S$ is the PVT generated by an homogeneous stationary Poisson. Then $\Lambda$ is stabilizing and essentially asymptotically connected for the following stabilization field:
$$\forall x \in \mathbb{R}^{2}, \, R_x = \inf \lbrace \lVert x - X_{S,i} \rVert , \, X_{S,i} \in X_{S} \rbrace,$$
where $X_S$ is the PPP having generated $S$.
\end{Proposition}

We will need the following notation for convenience:

\begin{Notation}
Assume that $\Xi$ is a stabilizing random measure for the stabilization field $R = \lbrace R_x \rbrace _{x \in \mathbb{R}^{2}}.$ Then, for $x \in \mathbb{R}^{2}$ and $n \in \mathbb{N}\setminus \lbrace 0 \rbrace$ we denote:
$$R(Q_n(x)) \coloneqq \sup_{y \in Q_n(x) \cap \mathbb{Q}^{2}} R_y$$
\end{Notation}

Finally, we will adapt the usual definitions of openness and closedness of crossroads and parts of street segments (possibly the whole street segments themselves) in our model as follows:

\begin{Definition}[Open/Closed crossroad]
\label{Def.open/crossroad}
Say a crossroad $V_{i} \in V$ is \emph{open} if it is an atom of the point process $Y$, i.e. $Y(\lbrace V_i \rbrace) = 1$, or, equivalently, $U_i \leq p$. \\ Say $V_{i}$ is \emph{closed} if it is not open, i.e. $Y(\lbrace V_i \rbrace) = 0$, or, equivalently, $U_i > p$.
\end{Definition}

\begin{Definition}[Open/Closed street segment]
\label{Def.open/closed/subcritical}
Let $e \in E$ be a street segment and let $\emptyset \neq s \subseteq e$ be a non-empty subset of $e$.\\ Say $s$ is \emph{open} if either of the two following set of conditions are satisfied:
\begin{enumerate}
\item $\vert s \vert \leq r$
\vspace{.2 cm}
\item[]\textbf{OR}
\vspace{.2 cm}
\item $\left\{
\begin{array}{l}
\vert s \vert > r \\
\forall c \subset s, \, (\vert c \vert = r \; \text{and} \; c  \; \text{topologically closed} )\Rightarrow X^{\lambda}(c) \geq 1
\end{array}
\right.$
\end{enumerate}
Say $s$ is \emph{closed} if $s$ is not open, i.e.: \\
 $\left\{
\begin{array}{l}
\vert s \vert > r \\
\exists \, c \subset s, \; \text{such that} \, \vert c \vert = r \; \text{and} \; c  \; \text{topologically closed} \; \text{and} \,  X^{\lambda}(c) = 0
\end{array}
\right.$
\end{Definition}

\section{Results}
Our results concern the phase transition of the connectivity graph corresponding to the model presented in Section~\ref{Ss.NetworkModel} as well as a minimality condition on $p$ to make percolation to occur possible. Namely, we have the following:

\begin{Theorem}[Minimality condition on $p$]
\label{Thm.minimality}
Let $p^{*}$ be the Voronoi percolation site threshold (a theoretical estimate is given in~\cite{neher2008topological} and a numerical one is given in~\cite{becker_percolation_2009}). Let $p < p^*$. Then, for all $\lambda > 0$ and $r >0$, the connectivity graph $\mathcal{G}_{r, \lambda, p}$ does not percolate with probability 1.
\end{Theorem}

Concerning the sub-critical phase, we have the following:
\begin{Theorem}[Existence of a non-trivial sub-critical phase]
\label{Thm.subcritical}
Assume $p=1$. Then there exists $r_{0} > 0$ such that, whenever $r \leq r_0$, we have:
\begin{equation*}
    \lambda_c(r) \coloneqq \inf \lbrace \lambda > 0 : \, \mathbb{P}(\mathcal{G}_{r, \lambda, 1} \; \text{percolates}) > 0 \rbrace > 0
\end{equation*}
In other words, there exists a subcritical phase for percolation of the connectivity graph when all crossroads are equipped with a relay if the connection radius is not too large.
\end{Theorem}

A straightforward consequence of Theorem~\ref{Thm.subcritical} is the following:

\begin{Corollary}
\label{Coroll.subcritical}
Let $r_0$ be defined as in Theorem~\ref{Thm.subcritical}. Then, whenever $r \leq r_{0}$, for all $p \in \left(p^*,1\right]$, we have:
\begin{equation*}
    \inf \lbrace \lambda > 0 : \, \mathbb{P}(\mathcal{G}_{r, \lambda, p} \; \text{percolates}) > 0 \rbrace > 0
\end{equation*}
\end{Corollary}

Finally, we also were able to get a matching super-criticality result:

\begin{Theorem}[Existence of a non-trivial super-critical phase]
\label{Thm.supercritical}
For sufficiently large and finite $\lambda$ and sufficiently large $p > p^*$, $\mathcal{G}_{r,\lambda,p}$ percolates, i.e. there exists a non-trivial super-critical phase.
\end{Theorem}

While Theorem~\ref{Thm.minimality} is quite straightforward, Theorems~\ref{Thm.subcritical} and ~\ref{Thm.supercritical} require the use of renormalization techniques similar to the ones exposed in \cite{hirsch_continuum_2017} and the domination by product measures theorem \cite[Theorem 0.0]{liggett_domination_1997}. \\
We will carry the proofs of the former theorems in the rest of this section.

\subsection{Proof of Theorem~\ref{Thm.minimality}}\mbox{}\\
\indent Let $p^*$ denote the usual Poisson-Voronoi site percolation threshold, as defined in ~\cite{becker_percolation_2009,neher2008topological}. It is known that $p \in \left(0,1\right)$. Note that by stationarity, $p^*$ is independent of the intensity of the PPP generating the considered PVT. Hence, consider site percolation on the PVT $S$ with parameter $p \in \left(0,1\right)$ and denote by $\mathcal{G}_p$ the associated graph. \\
Now, note that for given $p \in \left[0,1\right]$, all $\lambda >0$ and $r>0$, the edge-set of the connectivity graph of our model $\mathcal{G}_{r,\lambda,p}$ is a subset of the edge-set of $\mathcal{G}_p$. 
\\Hence: $\forall \, \lambda >0, \forall \, r>0, \, \mathcal{G}_{p} \ \text{does not percolate} \Rightarrow \, \mathcal{G}_{r, \lambda, p} \ \text{does not percolate}$.
Now, by definition of $p^*$, $\forall \, p < p^*, \, \mathcal{G}_p$ does not percolate. This concludes the proof of Theorem~\ref{Thm.minimality}. \qed
\\
\subsection{Proof of Theorem~\ref{Thm.subcritical}}\mbox{}\\
Proving Theorem~\ref{Thm.subcritical} is equivalent to proving that there exists $r_0$ such that whenever $r \leq r_0$, $\mathcal{G}_{r, \lambda, 1}$ does not percolate if $\lambda$ is sufficiently small but positive. \\
As customary in any continuum percolation problem and as has been done in \cite{hirsch_continuum_2017}, we will use a renormalization argument and introduce a discrete percolation model constructed in such a way that if the discrete model does not percolate, then neither does $\mathcal{G}_{r, \lambda,1}$. Proving the absence of percolation of the discrete model will then be done via appealing to \cite[Theorem 0.0]{liggett_domination_1997}. \\
\\
To this end, for $n \geq 1$, say a site $z \in \mathbb{Z}^{2}$ is \emph{n-good} if the following conditions are satisfied: 
\begin{enumerate}
\item $R(Q_n(nz)) < n$
\item $\forall e \in E, \, s_{z,e} \coloneqq e \cap Q_n(nz) \; \text{is closed}$
\end{enumerate}
Say a site $z \in \mathbb{Z}^{2}$ is \emph{$n$-bad} if it is not $n$-good. \\
\\
Our first claim is the following:
\begin{Lemma}
\label{Claim1.subcritical}
Percolation of $\mathcal{G}_{r, \lambda,1}$ implies percolation of the process of $n$-bad sites.
\end{Lemma}
\begin{Proof}
Assume $\mathcal{G}_{r,\lambda,1}$ percolates and denote by $\mathcal{C}$ a giant component of $\mathcal{G}_{r, \lambda, 1}$. Since $\mathcal{C}$ is infinite, we have: $\# \lbrace z : \mathcal{C} \cap Q_n(nz) \neq \emptyset \rbrace = + \infty$.
\\ Denote $ \lbrace z : \mathcal{C} \cap Q_n(nz) \neq \emptyset \rbrace \coloneqq \lbrace z_i, i \geq 1 \rbrace$. Note that $\mathcal{C}$ is made of open street segments and open crossroads. Therefore, since $Q_n(nz_i)$ is crossed by $\mathcal{C}$ for all $i \geq 1$, $Q_n(nz_i)$ has to be crossed by some open non-empty $s_e$, for some $e \in  E$. Hence $z_i$ is $n$-bad, and $\lbrace z_i, i \geq 1 \rbrace$ is a infinite component of $n$-bad sites. Now, since $\mathcal{C}$ is connected, then so is $\lbrace z_i, i \geq 1 \rbrace$. Hence, the process of $n$-bad sites percolates. 
\end{Proof}
Therefore, it suffices to prove that the process of $n$-bad sites does not percolate if $\lambda$ and $r$ are sufficiently small. This will be done via appealing to \cite[Theorem 0.0]{liggett_domination_1997}. \\
The conditions of \cite[Theorem 0.0]{liggett_domination_1997} are valid for so-called $k$-dependent random fields:
\begin{Definition}
Let $ \mathbf{X}=(X_s)_{s \in \mathbb{Z}^{d}}$ be a discrete random field. Let $k \geq 1$. Then $\mathbf{X}$ is said to be $k$-dependent if for all $p \geq 1$ and all $\psi = \lbrace s_{1}, \ldots s_{p} \rbrace \subset \mathbb{Z}^{d}$ finite with the property that $\forall i \neq j, \lVert s_{i} - s_{j} \rVert_{\infty} > k $, the random variables $(X_{s_{i}})_{1 \leq i \leq p}$ are independent.
\end{Definition}

As we shall see thereafter, the process of $n$-bad sites previously defined is $3$-dependent:

\begin{Lemma}
\label{Claim2.subcritical}
For $z \in \mathbb{Z}^{2}$, set $\zeta_{z} \coloneqq \mathbbm{1}\lbrace z \, \text{is $n$-bad} \rbrace$. Then $(\zeta_{z})_{z \in \mathbb{Z}^{2}}$ is a $3$-dependent random field.
\end{Lemma}
\begin{Proof}
As a starting point, note that $\forall z \in \mathbb{Z}^{2}, \zeta_{z} = 1 - \mathbbm{1} \lbrace z \, \text{is $n$-good} \rbrace$. It is therefore equivalent to prove that the process of $n$-good sites is $3$-dependent. \\
For $z \in \mathbb{Z}^{2}$, set $\xi_{z} = \mathbbm{1} \lbrace z \, \text{is $n$-good} \rbrace$. Let $\psi = \lbrace z_{1}, \ldots z_{p} \rbrace \subset \mathbb{Z}^{2}$ be such that $\forall i \neq j, \lVert z_i - z_j \rVert_{\infty} > 3$.
We want to show that the random variables $(\xi_{z_i})_{1\leq i \leq p}$ are independent. Equivalently, since we are dealing with indicator functions, this amounts to showing that: 
$$\mathbb{E}\left(\prod_{i=1}^{p} \xi_{z_i} \right) = \prod_{i=1}^{p} \mathbb{E}(\xi_{z_i})$$
Now, we have:

\begin{align}
  \nonumber \mathbb{E}\left(\prod_{i=1}^{p} \xi_{z_i} \right) &= \mathbb{E}\left[\mathbb{E}\left(\prod_{i=1}^{p} \xi_{z_i} \Bigg\vert \Lambda \right)\right] \\ \label{eq2} &= \mathbb{E}\Bigg[\mathbb{E}\Bigg(\prod_{i=1}^{p}\mathbbm{1}\lbrace R(Q_n(nz_i)) < n \rbrace \prod_{i=1}^{p} \mathbbm{1} \lbrace \forall \, e \in E, \, s_{z_i,e} \, \text{is closed} \rbrace  \Bigg\vert \Lambda \Bigg) \Bigg] 
\end{align}

By $\Lambda$-measurability of the random variables $\lbrace R_x \rbrace_{x \in \mathbb{R}^2}$, we can take the indicators of the form $\mathbbm{1} \lbrace R(Q_n(nz_i)) < n \rbrace$, $1 \leq i \leq p$ out of the conditional expectation given $\Lambda$ in~\eqref{eq2}. This yields:

\begin{align}
 \label{eq3} \mathbb{E}\left(\prod_{i=1}^{p} \xi_{z_i} \right) &= \mathbb{E}\Bigg[\prod_{i=1}^{p} \mathbbm{1} \lbrace R(Q_n(nz_i) <n \rbrace \mathbb{E}\Bigg( \prod_{i=1}^{p} \mathbbm{1} \lbrace \forall \, e \in E, \, s_{z_i,e} \, \text{is closed} \rbrace  \Bigg\vert \Lambda \Bigg) \Bigg] 
\end{align}
For $1 \leq i \leq p$, set $A_{z_i} \coloneqq \lbrace \forall \, e \in E, \, s_{z_i,e} \, \text{is closed} \rbrace \, , 1 \leq i \leq p$. According to Definition~\ref{Def.open/closed/subcritical}, for a given $1 \leq i \leq p$, the event $A_{z_i}$ only depends on the configuration of the random measure $\Lambda$ and of the Cox process $X^{\lambda}$ inside the square $Q_n(nz_i)$. Therefore, given $\Lambda$, the former events only depend on $X^{\lambda} \cap Q_n(nz_i)$ for $1 \leq i \leq p$. Since we have $\forall i \neq j, \lVert z_i - z_j \rVert_{\infty} > 3$, then the squares $Q_n(nz_i) = $ are disjoint. Now, we have that $X^{\lambda}$ is a Poisson Point Process given $\Lambda$. Thus, by Poisson independence property, the events $(A_{z_i})_{1 \leq i \leq p}$ are conditionally independent given $\Lambda$. Hence~\eqref{eq3} yields:
\begin{align}
 \label{eq4} \mathbb{E}\left(\prod_{i=1}^{p} \xi_{z_i} \right) &= \mathbb{E}\Bigg[\prod_{i=1}^{p} \mathbbm{1} \lbrace R(Q_n(nz_i) <n \rbrace \prod_{i=1}^{p} \mathbb{E}\Bigg(  \mathbbm{1} \lbrace \forall \, e \in E, \, s_{z_i,e} \, \text{is closed} \rbrace  \Bigg\vert \Lambda \Bigg) \Bigg] 
\end{align}
Now, setting, $f(\Lambda_{Q_n(x)}) \coloneqq \mathbb{E}\Bigg(  \mathbbm{1} \lbrace \forall \, e \in E, \, s_{x,e} \, \text{is closed} \rbrace  \Bigg\vert \Lambda \Bigg)$, a deterministic, bounded and measurable function of $\Lambda_{Q_n(x)}$, we are left with:
\begin{align}
    \mathbb{E}\left(\prod_{i=1}^{p} \xi_{z_i} \right) &= \mathbb{E}\left( \prod_{i=1}^{p} f(\Lambda_{Q_n(nz_i)}) \mathbbm{1}\lbrace R(Q_n(nz_i)) <n \rbrace \right) \label{eq5}
\end{align}
Now, the set $\varphi \coloneqq \lbrace nz_1,\ldots,nz_p \rbrace \subset \mathbb{R}^{d}$ is finite and satisfies: $$\forall i \neq j, \lVert nz_i - nz_j \rVert_{\infty} > 3n$$ Since the infinite norm is always upper bounded by the Euclidean norm, we have $\forall i \neq j, \lVert nz_i - nz_j \rVert_{2} > 3n $, and so $\varphi$ satisfies:
$$\forall x \in \varphi, \, \text{dist}(x, \varphi \setminus \lbrace x \rbrace) > 3n$$
We can therefore apply the condition (3) in the definition of stabilization (Definition~\ref{Def.stabilizing})  to get that the random variables appearing in the right-hand side of~\eqref{eq5} are independent. Hence:

\begin{align}
    \nonumber \mathbb{E}\left(\prod_{i=1}^{p} \xi_{z_i} \right) &= \prod_{i=1}^{p} \mathbb{E}\left( f(\Lambda_{Q_n(nz_i)})\mathbbm{1}\lbrace R(Q_n(nz_i)) <n \rbrace \right) \\\nonumber &= \prod_{i=1}^{p} \mathbb{E}\left[  \mathbb{E}\Bigg(  \mathbbm{1} \lbrace \forall \, e \in E, \, s_{z_i,e} \, \text{is closed} \rbrace  \Bigg\vert \Lambda \Bigg) \mathbbm{1}\lbrace R(Q_n(nz_i)) <n \rbrace \right] \nonumber \\ &= \prod_{i=1}^{p} \mathbb{E}\left[  \mathbb{E}\Bigg(  \mathbbm{1} \lbrace \forall \, e \in E, \, s_{z_i,e} \, \text{is closed} \rbrace \mathbbm{1}\lbrace R(Q_n(nz_i)) <n \rbrace  \Bigg\vert \Lambda \Bigg)   \right]  \label{eq6} \intertext{(where we have used $\Lambda$-measurability of the $R$'s in~\eqref{eq6})} \nonumber  \\ \nonumber &= \prod_{i=1}^{p} \mathbb{E}\left[   \mathbbm{1} \lbrace \forall \, e \in E, \, s_{z_i,e} \, \text{is closed} \rbrace \mathbbm{1}\lbrace R(Q_n(nz_i)) <n \rbrace \right] \\ \nonumber &\eqqcolon \prod_{i=1}^{p} \mathbb{E}(\xi_{z_i}) 
\end{align}
This concludes the proof of the lemma.
\end{Proof}
Now that we have proven that the process of $n$-bad sites is $3$-dependent, in order to apply \cite[Theorem 0.0]{liggett_domination_1997}, it remains to prove that the probability for a site $z \in \mathbb{Z}^{d}$ to be $n$-bad is sufficiently small when $\lambda$ and $r$ are chosen sufficiently small. Equivalently, as has been done in \cite{hirsch_continuum_2017}, we need to prove the following:
\begin{equation*}
    \limsup_{n \uparrow \infty}\limsup_{r \downarrow 0}\limsup_{\lambda \downarrow 0} \mathbb{P}(z \, \text{is $n$-bad}) = 0
\end{equation*}
By stationarity, it suffices to prove that: 
\begin{equation*}
\limsup_{n \uparrow \infty}\limsup_{r \downarrow 0}\limsup_{\lambda \downarrow 0} \mathbb{P}(0 \, \text{is $n$-bad}) = 0
\end{equation*}
Now, note that we have:
\begin{align*}
    \mathbb{P}(0 \, \text{is $n$-bad}) &= \mathbb{P}\Big(\lbrace R(Q_n) \geq n \rbrace \cup \lbrace \exists \, e \in E, \, e \cap Q_n \, \text{is open} \rbrace \Big) \\ & \leq \mathbb{P}\left( R(Q_n) \geq n \right) + \mathbb{P}\left(  \exists \, e \in E, \, e \cap Q_n \, \text{is open} \right)
\end{align*}
On the one hand, we note that the quantity $R(Q_n)$ does not depend on $\lambda$ and $r$ and by Definition~\ref{Def.stabilizing}, we have $\displaystyle \lim_{n \uparrow \infty} \mathbb{P}(R(Q_n) \geq n)  = 0$. \\
Therefore, it remains to deal with the quantity $\mathbb{P}\left(  \exists \, e \in E, \, e \cap Q_n \, \text{is open} \right)$. We have:
\begin{align}
    \nonumber \mathbb{P}\left(\exists \, e \in E, \, e \cap Q_n \, \text{is open} \right) &= 1 - \mathbb{P}\left(  \forall \, e \in E, \, e \cap Q_n \, \text{is open} \right) \\ \nonumber &= 1 - \mathbb{E}\left( \prod_{e \in E \, : \, e \cap Q_n \neq \emptyset} \mathbbm{1}\lbrace \text{$e \cap Q_n$ is open} \rbrace \right) \\ \nonumber &= 1 - \mathbb{E}\left[\mathbb{E}\left( \prod_{e \in E \, : e \cap Q_n \neq \emptyset} \mathbbm{1}\lbrace \text{$e \cap Q_n$ is closed} \rbrace  \Bigg\vert \Lambda \right) \right] 
\end{align}
For $e \in E$, denote:
\begin{gather*}
    s_e \coloneqq e \cap Q_n \\ A_e \coloneqq \lbrace \exists \, c \subset s_e, \, \text{$c$ topologically closed}, \vert c \vert = r, X^{\lambda}(c) = 0 \rbrace
\end{gather*}
We have: $\mathbbm{1}\lbrace \text{$s_e$ is closed} \rbrace = \mathbbm{1}\lbrace \vert s_e \vert > r \rbrace \mathbbm{1}\lbrace A_e \rbrace $. Thus:

\begin{align}
     \nonumber \mathbb{P}\left(\exists \, e \in E, \, e \cap Q_n \, \text{is open} \right) &= 1 - \mathbb{E}\left[\mathbb{E}\left( \prod_{e \in E \, : s_e \neq \emptyset} \mathbbm{1}\lbrace \vert s_e \vert > r \rbrace \mathbbm{1}\lbrace A_e \rbrace  \Bigg\vert \Lambda \right) \right] \\ \nonumber &= 1 - \mathbb{E}\left[\prod_{e \in E : s_e \neq \emptyset} \mathbbm{1}\lbrace \vert s_e \vert > r \rbrace\mathbb{E}\left( \prod_{e \in E \, : s_e \neq \emptyset}  \mathbbm{1}\lbrace A_e \rbrace  \Bigg\vert \Lambda \right) \right],  \intertext{by $\Lambda$-measurability of the events $\lbrace \vert s_e > r \vert \rbrace$,\, $e \in E$.} \nonumber
\end{align}
Given $\Lambda$, $X^{\lambda}$ has the distribution of a Poisson point process with mean measure $\lambda \Lambda$ and $A_e$ only depends on $X^{\lambda} \cap e$ once $e$ is fixed. Therefore, given $\Lambda$, the events $\lbrace A_e \, : e \in E \rbrace$ depend on the number of Cox points on distinct edges and so, by the Poisson independence property, these events are conditionally independent given $\Lambda$. Thus: 

\begin{align}
     \nonumber \mathbb{P}\left(\exists \, e \in E, \, e \cap Q_n \, \text{is open} \right) &= 1 - \mathbb{E}\left[\prod_{e \in E : s_e \neq \emptyset} \mathbbm{1}\lbrace \vert s_e \vert > r \rbrace\mathbb{E}\left( \prod_{e \in E \, : s_e \neq \emptyset}  \mathbbm{1}\lbrace A_e \rbrace  \Bigg\vert \Lambda \right) \right] \\ \nonumber &= 1 - \mathbb{E}\left[\prod_{e \in E : s_e \neq \emptyset} \mathbbm{1}\lbrace \vert s_e \vert > r \rbrace \prod_{e \in E \, : s_e \neq \emptyset}\mathbb{E}\left(   \mathbbm{1}\lbrace A_e \rbrace  \vert \Lambda \right) \right] \\ \label{eq7} &= 1 - \mathbb{E}\left[\prod_{e \in E : s_e \neq \emptyset} \mathbbm{1}\lbrace \vert s_e \vert > r \rbrace \prod_{e \in E \, : s_e \neq \emptyset}\mathbb{P}\left( A_e  \vert \Lambda \right) \right] 
\end{align}
For all $e \in E$ such that $s_e \neq \emptyset$, we have that $0 \leq \mathbb{P}(A_e \vert \Lambda) \leq 1$, and moreover the event $A_e$ is increasing in $\lambda$ with $\displaystyle \lim_{\lambda \downarrow 0} \mathbbm{1}\lbrace A_e \rbrace = 1 \: \text{a.s.}$. So, by monotone convergence, we have that for all $e \in E$ such that $s_e \neq \emptyset$, $\displaystyle \lim_{\lambda \downarrow 0} \mathbb{P}(A_e \vert \Lambda) = 1 \: \text{a.s.}$. As a matter of fact:
\begin{equation*}
    \limsup_{\lambda \downarrow 0} \prod_{e \in E : s_e \neq \emptyset} \mathbbm{1}\lbrace \vert s_e \vert > r \rbrace \prod_{e \in E \, : s_e \neq \emptyset}\mathbb{P}\left(    A_e  \vert \Lambda \right) = \prod_{e \in E : s_e \neq \emptyset} \mathbbm{1}\lbrace \vert s_e \vert > r \rbrace \quad \text{a.s.}
\end{equation*}
Noting that $\displaystyle \Bigg\lvert \prod_{e \in E : s_e \neq \emptyset} \mathbbm{1}\lbrace \vert s_e \vert > r \rbrace \prod_{e \in E \, : s_e \neq \emptyset}\mathbb{P}\left(    A_e  \vert \Lambda \right) \Bigg\rvert \leq 1$, we can apply dominated convergence in~\eqref{eq7}, we get:
\begin{align}
   \nonumber  \limsup_{\lambda \downarrow 0} \mathbb{P}\left(\exists \, e \in E, \, e \cap Q_n \, \text{is open} \right) &= 1 - \mathbb{E}\left(\prod_{e \in E : s_e \neq \emptyset} \mathbbm{1}\lbrace \vert s_e \vert > r \rbrace \right) \\ \label{eq8} &= 1 - \mathbb{E}\left(\prod_{e \in E} \mathbbm{1}\lbrace \vert e \cap Q_n \vert > r \rbrace \mathbbm{1} \lbrace e \cap Q_n \neq \emptyset \rbrace \right)
\end{align}
Again, by monotone convergence, we have for all $e \in E$:
\begin{align*}
\lim_{r \downarrow 0} \mathbbm{1}\lbrace \vert e \cap Q_n \vert > r \rbrace \mathbbm{1}\lbrace e \cap Q_n \neq \emptyset \rbrace &= \mathbbm{1}\lbrace \vert e \cap Q_n \vert > 0 \rbrace \mathbbm{1}\lbrace e \cap Q_n \neq \emptyset \rbrace \quad \text{a.s.} \\ &= \mathbbm{1}\lbrace e \cap Q_n \neq \emptyset \rbrace \quad \text{a.s.}
\end{align*} 
So, applying dominated convergence again in~\eqref{eq8} we get:
\begin{equation}
\label{eq9}
   \limsup_{r \downarrow \infty}\limsup_{\lambda \downarrow 0} \mathbb{P}\left(\exists \, e \in E, \, e \cap Q_n \, \text{is open} \right)=1-\mathbb{E}\left(\prod_{e \in E } \mathbbm{1}\lbrace e \cap Q_n \neq \emptyset \rbrace \right)
\end{equation}
By monotone convergence, $\displaystyle \lim_{n \uparrow \infty}\mathbbm{1}\lbrace e \cap Q_n \neq \emptyset \rbrace = 1 \quad \text{a.s.}$ for all $e \in E$. Therefore, by dominated convergence in~\eqref{eq9}, we obtain:
\begin{equation}
    \label{eq10}
    \limsup_{n \uparrow \infty}\limsup_{r \downarrow 0}\limsup_{\lambda \downarrow 0} \mathbb{P}\left(\exists \, e \in E, \, e \cap Q_n \, \text{is open} \right) = 1-1 = 0
\end{equation}
To conclude, we have: 
\begin{equation*}
    0 \leq \mathbb{P}(0 \, \text{is $n$-bad}) \leq \mathbb{P}(R(Q_n) \geq n) + \mathbb{P}(\exists \, e \in E, \, e \cap Q_n \, \text{is open})
\end{equation*}
Using Definition~\ref{Def.stabilizing} on  and~\eqref{eq10}, we finally get:
\begin{equation*}
     \limsup_{n \uparrow \infty}\limsup_{r \downarrow 0}\limsup_{\lambda \downarrow 0} \mathbb{P}(0 \, \text{is $n$-bad}) =0
\end{equation*}
Hence, by \cite[Theorem 0.0]{liggett_domination_1997}, the process of $n$-bad sites is stochastically dominated from above by a sub-critical Bernoulli process when $\lambda$ and $r$ are sufficiently small. In particular, the process of $n$-bad sites cannot percolate when $\lambda$ and $r$ are sufficiently small. In other words, there exists $r_0 >0$ such that $\lambda_c(r) > 0$. This concludes the proof of Theorem~\ref{Thm.subcritical}. \qed

\subsection{Proof of Corollary~\ref{Coroll.subcritical}}\mbox{}\\
Let $r_0$ be defined as in Theorem~\ref{Thm.subcritical}. Let $r \leq r_0$ and $p \in \left( p^{*},A \right]$. For fixed $\lambda > 0$, there are obviously fewer possible connections in the connectivity graph $\mathcal{G}_{r,\lambda,p}$ than in $\mathcal{G}_{r,\lambda,1}$. In other words, the vertex-set (resp. edge-set) of $\mathcal{G}_{r,\lambda,p}$ is a subset of the vertex-set (resp. edge-set) of $\mathcal{G}_{r, \lambda,1}$. Thus we have:
\begin{equation*}
    \mathcal{G}_{r, \lambda,p} \, \text{percolates} \,  \Rightarrow \mathcal{G}_{r, \lambda,1} \, \text{percolates}
\end{equation*}
Therefore:
\begin{equation*}
    \inf \lbrace \lambda > 0 : \, \mathbb{P}(\mathcal{G}_{r, \lambda, p} \; \text{percolates}) > 0 \rbrace >   \underbrace{\inf \lbrace \lambda > 0 : \, \mathbb{P}(\mathcal{G}_{r, \lambda, 1} \; \text{percolates}) > 0 \rbrace}_{ \eqqcolon \lambda_c(r)}
\end{equation*}
By Theorem~\ref{Thm.subcritical}, we have that $\lambda_c(r) > 0$ whenever $r \leq r_0$. This concludes the proof of Corollary~\ref{Coroll.subcritical}. \qed

\subsection{Proof of Theorem~\ref{Thm.supercritical}}\mbox{}\\
As in the proof of Theorem~\ref{Thm.subcritical}, we will use a renormalization argument to prove that $\mathcal{G}_{r, \lambda, p}$ percolates if $\lambda$ and $p \in \left(p^{*},1\right]$ are chosen sufficiently large. \\
To this end, let us first define a discrete model in such a way that percolation of the discrete model will imply percolation of the continuum one. For $n \geq 1$, say a site $z \in \mathbb{Z}^{2}$ is $n$-good if the following conditions are satisfied:
\begin{enumerate}
\item $R(Q_{6n}(nz)) < n/2$
\item $E \cap Q_n(nz) \neq \emptyset$, i.e. the square $Q_n(nz)$ contains a \emph{full} street segment (not just a subset of a street segment)
\item There exists $e \in E \cap Q_n(nz)$ such that $e$ is open, in the sense of Definition~\ref{Def.open/closed/subcritical}. In other words, there exists an open edge which is fully included in the square $Q_n(nz)$.
\item Every two open edges $e,e' \in E \cap Q_{3n}(nz)$ are connected by a path in $\mathcal{G}_{r,\lambda,1}\cap Q_{6n}(nz)$, i.e. $e$ and $e'$ are connected by a path of the connectivity graph staying inside of the larger square $Q_{6n}(nz)$ if all crossroads are open.
\item $Y(Q_{6n}(nz)) = \#(V \cap Q_{6n}(nz))$, i.e. all crossroads in $Q_{6n}(nz)$ are open, in the sense of Definition~\ref{Def.open/crossroad}.
\end{enumerate}
As in the proof of Theorem~\ref{Thm.subcritical}, for $n \geq 1$, say a site $z \in \mathbb{Z}^{2}$ is $n$-bad if it is not $n$-good. \\

The $n$-good sites have been defined so as to satisfy the following:
\begin{Lemma}
\label{Claim1.supercritical}
Percolation of the process of $n$-good sites implies percolation of the connectivity graph $\mathcal{G}_{r, \lambda,p}$.
\end{Lemma}
\begin{Proof}
Let $\mathcal{C}$ be an infinite connected component of $n$-good sites. Consider a finite path $\lbrace z_0, \ldots, z_q \rbrace \subset \mathcal{C}$ of $n$-good sites $(q \geq 1)$. Then, by condition $(2)$ in the definition of $n$-goodness, we can find an open edge $e_j \in E \cap Q_n(nz_j)$, for each $0 \leq j \leq q$. Let $0 \leq j \leq q-1$. Then $z_j = (a,b)$ for some $a \in \mathbb{Z}$ and $b \in \mathbb{Z}$. Since $\mathcal{C}$ is connected in $\mathbb{Z}^{2}$, we have $z_{j+1} \in \lbrace (a \pm 1, b), (a, b \pm 1) \rbrace$. By symmetry, we can assume $z_{j+1} = (a+1,b)$. Thus:
\begin{gather*}
Q_n(nz_j) = \left[na-n/2,na+n/2\right] \times \left[nb-n/2,nb+n/2\right]\\ Q_n(nz_{j+1}) = \left[na+n/2,na+3n/2\right] \times \left[nb-n/2,nb+n/2\right] \\
Q_{3n}(nz_j) = \left[na-3n/2,na+3n/2\right] \times \left[nb-3n/2,nb+3n/2\right] \\
Q_{6n}(nz_j) = \left[na-3n,na+3n\right] \times \left[nb-3n,nb+3n\right]
\end{gather*}
Therefore, we have $Q_{n}(nz_{j+1}) \subset Q_{3n}(nz_j)$ and so $e_{j+1} \in E \cap Q_{n}(nz_{j+1})$ implies $e_{j+1} \in E \cap Q_{3n}(nz_j)$. Since we also have $e_{j} \in E \cap Q_{n}(nz_j) \subset  E \cap Q_{3n}(nz_j)$ and that $e_j$ and $e_{j+1}$ are both open, by condition (4) in the definition of,$n$-goodness, $e_j$ and $e_{j+1}$ are connected by a path $\mathcal{L}$ in $\mathcal{G}_{r,\lambda,1} \cap Q_{6n}(nz_j)$. But since $z_j$ is an $n$-good site, by condition (5) in the definition of $n$-goodness, all crossroads inside of $Q_{6n}(nz_j)$ are open. Therefore, the path $\mathcal{L}$ also connects $e_j$ and $e_{j+1}$ in $\mathcal{G}_{r, \lambda,p} \cap Q_{6n}(nz_j)$. Iterating this process gives rise to an infinite connected component in $\mathcal{G}_{r,\lambda,p}$. This concludes the proof of Lemma~\ref{Claim1.supercritical}.
\end{Proof}
As a matter of fact, proving Theorem~\ref{Thm.supercritical} amounts to proving that the process of $n$-good sites percolates for sufficiently large $\lambda$ and $p$. \\
As has been done in the proof of Theorem~\ref{Thm.subcritical}, we will stochastically dominate the process of $n$-good sites by a Bernoulli process via the means of \cite[Theorem 0.0]{liggett_domination_1997}. To check that the conditions of this theorem are satisfied, we will need to use a slightly modified version of $n$-goodness more adapted to the use of the aforementioned theorem, as follows: \\

For $n \geq 1$, say a site $z \in \mathbb{Z}^{2}$ is weakly-$n$-good if the following conditions are satisfied:
\begin{enumerate}
\item[$\widetilde{(1)}$] $R(Q_{6n}(nz)) < 6n$
\item[(2)]  $E \cap Q_n(nz) \neq \emptyset$, i.e. the square $Q_n(nz)$ contains a \emph{full} street segment (not just a subset of a street segment)
\item[(3)]  There exists $e \in E \cap Q_n(nz)$ such that $e$ is open, in the sense of Definition~\ref{Def.open/closed/subcritical}. In other words, there exists an open edge which is fully included in the square $Q_n(nz)$.
\item[(4)]  Every two open edges $e,e' \in E \cap Q_{3n}(nz)$ are connected by a path in $\mathcal{G}_{r,\lambda,1}\cap Q_{6n}(nz)$, i.e. $e$ and $e'$ are connected by a path of the connectivity graph staying inside of the larger square $Q_{6n}(nz)$ if all crossroads are open.
\item[(5)]  $Y(Q_{6n}(nz)) = \#(V \cap Q_{6n}(nz))$, i.e. all crossroads in $Q_{6n}(nz)$ are open, in the sense of Definition~\ref{Def.open/crossroad}.
\end{enumerate}
Then the following is clear:
\begin{Lemma}
\label{Lemma1.almost}
$\forall \, z \in \mathbb{Z}^{2}$, $z$ is $n$-good $\Rightarrow \, z$ is weakly-$n$-good. Therefore, we have the following: $\forall \, z \in \mathbb{Z}^{2}, \: \mathbb{P}(\text{$z$ is $n$-good}) \leq \mathbb{P}(\text{$z$ is weakly-$n$-good})$
\end{Lemma} Moreover, since the condition (1) in the definition of $n$-goodness has not been used in the proof of Lemma~\ref{Claim1.supercritical}, the following is also clear:
\begin{Lemma}
\label{Lemma2.almost}
Percolation of the process of weakly-$n$-good sites implies percolation of the connectivity graph $\mathcal{G}_{r,\lambda,p}$.
\end{Lemma}
The reason why considering almost-$n$-good sites instead of $n$-good sites will turn out to be easier for our purposes is the following:
\begin{Lemma}
\label{Claim2.supercritical}
For $z \in \mathbb{Z}^{2}$, set $\xi_{z} \coloneqq \mathbbm{1}\lbrace z \, \text{is weakly-$n$-good} \rbrace$. Then $(\xi_{z})_{z \in \mathbb{Z}^{2}}$ is an $18$-dependent random field.
\end{Lemma}
\begin{Proof}
In the same way as in the proof of Lemma~\ref{Claim1.subcritical}, it suffices to prove that for all finite $\psi = \lbrace z_{1}, \ldots z_{p} \rbrace \subset \mathbb{Z}^{2}$ such that $\forall i \neq j, \lVert z_i - z_j \rVert_{\infty} > 18$, we have:
$$\mathbb{E}\left(\prod_{i=1}^{p} \xi_{z_i} \right) = \prod_{i=1}^{p} \mathbb{E}(\xi_{z_i})$$
Denote respectively by $\widetilde{(A_z)},(B_z),(C_z),(D_z),(F_z)$ the events $\widetilde{(1)},(2),(3),(4),(5)$ in the definition of weakly-$n$-goodness for $z \in \mathbb{Z}^{2}$. We thus have: 
$$\forall z \in \mathbb{Z}^{2}, \, \xi_{z} = \mathbbm{1}\lbrace \widetilde{(A_z)} \rbrace \mathbbm{1}\lbrace (B_z) \rbrace \mathbbm{1}\lbrace (C_z) \rbrace \mathbbm{1}\lbrace (D_z) \rbrace \mathbbm{1}\lbrace (F_z) \rbrace$$

Note first that whenever $z \in \mathbb{Z}^{2}$, the indicators $\mathbbm{1} \lbrace \widetilde{(A_z)} \rbrace$ and $\mathbbm{1} \lbrace (B_z) \rbrace$ are $\Lambda$-measurable. Thus, we have :
\begin{align}
\nonumber \mathbb{E}\left(\prod_{i=1}^{p} \xi_{z_i} \right) &= \mathbb{E}\left[\mathbb{E}\left(\prod_{i=1}^{p} \xi_{z_i} \Bigg\vert \Lambda \right)\right] 
\\ \nonumber &= \mathbb{E}\left[\mathbb{E}\left(\prod_{i=1}^{p} \mathbbm{1}\lbrace \widetilde{(A_{z_i})} \rbrace \mathbbm{1}\lbrace (B_{z_i}) \rbrace \mathbbm{1}\lbrace (C_{z_i}) \rbrace \mathbbm{1}\lbrace (D_{z_i}) \rbrace \mathbbm{1}\lbrace (F_{z_i}) \rbrace \Bigg\vert \Lambda \right)\right] 
\\ \nonumber &= \mathbb{E}\left[\prod_{i=1}^{p} \mathbbm{1}\lbrace \widetilde{(A_{z_i})} \rbrace \mathbbm{1}\lbrace (B_{z_i})\rbrace\mathbb{E}\left(  \prod_{i=1}^{p}\mathbbm{1}\lbrace (C_{z_i}) \cap (D_{z_i}) \cap (F_{z_i})  \rbrace \Bigg\vert \Lambda \right)\right] 
\end{align}
Now, note that conditioned on $\Lambda$, for each $1 \leq i \leq p$, the event $ (C_{z_i}) \cap (D_{z_i}) \cap (F_{z_i}) $ only depends on the following:
\begin{itemize}
\item The configuration of $X^{\lambda}$ inside of the square $Q_{6n}(nz_i)$
\item The configuration of $Y$ inside of the square $Q_{6n}(nz_i)$
\end{itemize}
Since $\psi$ satisfies $\forall i \neq j, \lVert z_i - z_j \rVert_{\infty} > 18$, then we have $\forall i \neq j, \lVert nz_i - nz_j \rVert_{\infty} > 18n$. As a matter of fact, the squares $\lbrace Q_{6n}(nz_i)) \, : \, 1 \leq i \leq p\rbrace$ are disjoint, i.e. $$\forall i \neq j, Q_{6n}(nz_i) \cap Q_{6n}(nz_j) = \emptyset$$
As a matter of fact, the events $\lbrace (C_{z_i}) \cap (D_{z_i}) \cap (F_{z_i}) \, : \, 1 \leq i \leq p \rbrace$ depend on disjoint Borel sets. 
%so, by the independence property of Z
Now, conditioned on $\Lambda$, $X^{\lambda}$ has the distribution of a Poisson point process, $Y$ has the distribution of a Bernoulli point process and $X^{\lambda} \indep Y \, \vert \Lambda \,$. By the independence property for Poisson and Bernoulli point processes, we have that the events $ \lbrace (C_{z_i}) \cap (D_{z_i}) \cap (F_{z_i}) \, : \, 1 \leq i \leq p \rbrace$ are conditionally independent given $\Lambda$. Hence:
\begin{align}
\nonumber \mathbb{E}\left(\prod_{i=1}^{p} \xi_{z_i} \right)  &= \mathbb{E}\left[\prod_{i=1}^{p} \mathbbm{1}\lbrace \widetilde{(A_{z_i})} \rbrace \mathbbm{1}\lbrace (B_{z_i})\rbrace\mathbb{E}\left(  \prod_{i=1}^{p}\mathbbm{1}\lbrace (C_{z_i}) \cap (D_{z_i}) \cap (F_{z_i})  \rbrace \Bigg\vert \Lambda \right)\right] 
\\ \nonumber &=  \mathbb{E}\left[\prod_{i=1}^{p} \mathbbm{1}\lbrace \widetilde{(A_{z_i})} \rbrace \mathbbm{1}\lbrace (B_{z_i})\rbrace \prod_{i=1}^{p}\mathbb{E}\left(  \mathbbm{1}\lbrace (C_{z_i}) \cap (D_{z_i}) \cap (F_{z_i})  \rbrace \Bigg\vert \Lambda \right)\right] 
\end{align}
Again, by $\Lambda$-measurability of the events $ \lbrace (B_{z_i}) \, : \, 1 \leq i \leq p \rbrace$, we can put the indicators $\mathbbm{1} \lbrace (B_{z_i}) \rbrace$ back into the conditional expectation given $\Lambda$, so as to get:
\begin{align}
\nonumber \mathbb{E}\left(\prod_{i=1}^{p} \xi_{z_i} \right)  &=  \mathbb{E}\left[\prod_{i=1}^{p} \mathbbm{1}\lbrace \widetilde{(A_{z_i})} \rbrace  \prod_{i=1}^{p}\mathbb{E}\left( \mathbbm{1}\lbrace (B_{z_i})\rbrace \mathbbm{1}\lbrace (C_{z_i}) \cap (D_{z_i}) \cap (F_{z_i})  \rbrace \Bigg\vert \Lambda \right)\right] 
\\ \nonumber &= \mathbb{E}\left[\prod_{i=1}^{p} \mathbbm{1}\lbrace \widetilde{(A_{z_i})} \rbrace  \prod_{i=1}^{p}\mathbb{E}\left( \mathbbm{1}\lbrace (B_{z_i}) \cap (C_{z_i}) \cap (D_{z_i}) \cap (F_{z_i})  \rbrace \Bigg\vert \Lambda \right)\right] 
\end{align}
Now, setting $f(\Lambda_{Q_{6n}(x)}) \coloneqq \mathbb{E}\left(\mathbbm{1}\lbrace (B_{x}) \cap (C_{x}) \cap (D_{x}) \cap (F_{x})  \rbrace \,  \vert \, \Lambda \right)$, a bounded measurable deterministic function of $\Lambda_{Q_{6n}(x)}$, we are left with:
\begin{align}
\nonumber \mathbb{E}\left(\prod_{i=1}^{p} \xi_{z_i} \right)  &=  \mathbb{E}\left[\prod_{i=1}^{p} \mathbbm{1}\lbrace \widetilde{(A_{z_i})} \rbrace  \prod_{i=1}^{p}\mathbb{E}\left( \mathbbm{1}\lbrace (B_{z_i})\rbrace \mathbbm{1}\lbrace (C_{z_i}) \cap (D_{z_i}) \cap (F_{z_i})  \rbrace \Bigg\vert \Lambda \right)\right] 
\\ \label{eq11} &= \mathbb{E}\left[\prod_{i=1}^{p} \mathbbm{1}\lbrace R(Q_{6n}(nz_i) < 6n \rbrace  f(\Lambda_{Q_{6n}(nz_i)})\right] 
\end{align}
Now, the set $\varphi \coloneqq \lbrace nz_1,\ldots,nz_p \rbrace \subset \mathbb{R}^{d}$ is finite and satisfies: $$\forall i \neq j, \lVert nz_i - nz_j \rVert_{\infty} > 18n $$ Since the infinite norm is always upper bounded by the Euclidean norm, we have $\forall i \neq j, \lVert nz_i - nz_j \rVert_{2} > 18n  $, and so $\varphi$ satisfies:
$$\forall x \in \varphi, \, \text{dist}(x, \varphi \setminus \lbrace x \rbrace) > 18n=3 \times 6n $$
We can therefore apply condition (3) in Definition~\ref{Def.stabilizing} (with $n$ replaced by $6n$) to get that the random variables appearing in the right-hand side of~\eqref{eq11} are independent. Hence:
\begin{align}
\label{eq12} \mathbb{E}\left(\prod_{i=1}^{p} \xi_{z_i} \right)  &= \prod_{i=1}^{p} \mathbb{E}\left[\mathbbm{1}\lbrace R(Q_{6n}(nz_i) < 6n \rbrace   f(\Lambda_{Q_{6n}(nz_i)})\right] 
\end{align}
Now, for each $1 \leq i \leq p$, note that:
\begin{gather*}
 \mathbb{E}\left[\mathbbm{1}\lbrace R(Q_{6n}(nz_i) < 6n \rbrace   f(\Lambda_{Q_{6n}(nz_i)})\right] 
\\  = \mathbb{E}\left[\mathbbm{1}\lbrace R(Q_{6n}(nz_i) < 6n \rbrace  \mathbb{E}\left(\mathbbm{1}\lbrace (B_{z_i}) \cap (C_{z_i}) \cap (D_{z_i}) \cap (F_{z_i})  \rbrace \,  \vert \, \Lambda \right)  \right]
\\ = \mathbb{E}\left[ \mathbb{E}\left(\mathbbm{1}\lbrace R(Q_{6n}(nz_i) < 6n \rbrace \mathbbm{1}\lbrace (B_{z_i}) \rbrace \mathbbm{1} \lbrace (C_{z_i}) \rbrace \mathbbm{1} \lbrace (D_{z_i}) \rbrace \mathbbm{1} \lbrace (F_{z_i})  \rbrace \,  \Bigg\vert \, \Lambda \right)  \right],
\\ \intertext{using $\Lambda$-measurability of $\mathbbm{1}\lbrace R(Q_{6n}(nz_i) < 6n \rbrace$}
\\ =  \mathbb{E}\left(\mathbbm{1}\lbrace R(Q_{6n}(nz_i) < 6n \rbrace \mathbbm{1}\lbrace (B_{z_i}) \rbrace \mathbbm{1} \lbrace (C_{z_i}) \rbrace \mathbbm{1} \lbrace (D_{z_i}) \rbrace \mathbbm{1} \lbrace (F_{z_i})  \rbrace \right)
\\ =  \mathbb{E}\left(\mathbbm{1}\lbrace (A_{z_i}) \rbrace \mathbbm{1}\lbrace (B_{z_i}) \rbrace \mathbbm{1} \lbrace (C_{z_i}) \rbrace \mathbbm{1} \lbrace (D_{z_i}) \rbrace \mathbbm{1} \lbrace (F_{z_i})  \rbrace \right) 
\\ \eqqcolon \mathbb{E}(\xi_{z_i})
\end{gather*}
Thus,~\eqref{eq12} yields:
\begin{equation*}
 \mathbb{E}\left(\prod_{i=1}^{p} \xi_{z_i} \right) = \prod_{i=1}^{p} \mathbb{E}(\xi_{z_i}),
\end{equation*}
as needed. This concludes the proof of Lemma~\ref{Claim2.supercritical}.
\end{Proof}
Our intention is to apply \cite[Theorem 0.0]{liggett_domination_1997} and stochastically dominate from below the process of weakly-$n$-good sites by a supercritical Bernoulli process. Since percolation of the weakly-$n$-good sites implies percolation of $\mathcal{G}_{r,\lambda,p}$, we will be done. \\

Now that we have proven that  the process of weakly-$n$-good sites is $18$-dependent, to apply \cite[Theorem 0.0]{liggett_domination_1997}, it remains to prove that the probability $z \in \mathbb{Z}^{d}$ to be weakly-$n$-good is arbitrarily large when $\lambda$ and $p$ are sufficiently large. Equivalently, as has been done in the proof of Theorem~\ref{Thm.subcritical}, we need to prove the following:
\begin{equation*}
    \limsup_{n \uparrow \infty}\limsup_{\lambda \uparrow \infty}\limsup_{p \uparrow 1} \mathbb{P}(z \, \text{is weakly-$n$-good}) = 1
\end{equation*}
By stationarity, it suffices to prove that: 
\begin{equation*}
\limsup_{n \uparrow \infty}\limsup_{\lambda \uparrow \infty}\limsup_{p \uparrow 1} \mathbb{P}(0 \, \text{is weakly-$n$-good}) = 1
\end{equation*}
Since $ \mathbb{P}(\text{$0$ is $n$-good}) \leq \mathbb{P}(\text{$0$ is weakly-$n$-good})$, it suffices to prove that:
\begin{equation*}
\limsup_{n \uparrow \infty}\limsup_{\lambda \uparrow \infty}\limsup_{p \uparrow 1} \mathbb{P}(0 \, \text{is $n$-good}) = 1
\end{equation*}
Or, equivalently:
\begin{equation*}
\limsup_{n \uparrow \infty}\limsup_{\lambda \uparrow \infty}\limsup_{p \uparrow 1} \mathbb{P}(0 \, \text{is $n$-bad}) = 0
\end{equation*}
For $z \in \mathbb{Z}^{2}$, denote respectively by $(A_z),(B_z), (C_z), (D_z), (F_z)$ the events (1), (2), (3), (4), (5) in the definition of $n$-goodness. Denote by $(A),(B),(C),(D),(F)$ the former events for $z=0$. We have:
\begin{align*}
    \mathbb{P}(0 \, \text{is $n$-bad}) &= \mathbb{P}(A^c \cup B^c \cup C^c \cup D^c \cup F^c)
\\ &= \mathbb{P}(A^c) + \mathbb{P}(A \cap B^c) + \mathbb{P}(A \cap B \cap C^c) + \mathbb{P}(A \cap B \cap C \cap F^c) \\ & \quad + \mathbb{P}(A \cap B \cap C \cap F \cap D^c)
\end{align*}
First, partitioning the square $Q_{6n}$ into $12^2=144$ subsquares $(Q_i)_{1 \leq i \leq 144}$ of side length $n/2$, we get:
\begin{align}
    \nonumber \mathbb{P}(A^c) &= \mathbb{P}(R(Q_{6n}) \geq n/2)
    \\ \nonumber & \leq  \mathbb{P}\left(\bigcup_{i=1}^{144} \left\lbrace  R(Q_i) \geq n/2 \right\rbrace \right)
    \\ \nonumber &= 144 \; \mathbb{P}(R(Q_{n/2}) \geq n/2) \qquad \text{by stationarity of the $R$'s} 
\end{align}
Therefore, by condition (2) of Definition~\ref{Def.stabilizing}, we get:
\begin{equation}
    \label{eq13}
    \lim_{n \uparrow \infty} \mathbb{P}(A^c) = 0
\end{equation}
Moreover, we have: 
\begin{align*}
    \mathbb{P}(A \cap B^c) & \leq \mathbb{P}(B^c)
    \\ & \eqqcolon \mathbb{P}(E \cap Q_n = \emptyset)
\end{align*}
By monotone convergence, we therefore get:
\begin{equation*}
    \lim_{n \uparrow \infty}\mathbb{P}(E \cap Q_n = \emptyset) = \mathbb{P}(E \cap \mathbb{R}^{2} = \emptyset) = 0,
\end{equation*}
since $\lambda_S > 0$ and so the Poisson-Voronoi tessellation $S$ is non-empty. Therefore, we have:
\begin{equation}
    \label{eq14}
    \lim_{n \uparrow \infty}\mathbb{P}(A \cap B^c) = 0
\end{equation}
Let's now deal with the quantity $\mathbb{P}(A \cap B \cap C^c)$. We have:
\begin{align*}
    \mathbb{P}(A \cap B \cap C^c) &\leq \mathbb{P}(B \cap C^c)
    \\ &= \mathbb{P}(\lbrace E \cap Q_n \neq \emptyset \rbrace \cap \lbrace \forall \, e \in E \cap Q_n, e \, \text{is closed} \rbrace)
    \\ &= \mathbb{E}\left( \mathbbm{1}\lbrace E \cap Q_n \neq \emptyset \rbrace \prod_{e \in E \cap Q_n} \mathbbm{1} \lbrace e \, \text{is closed} \rbrace \right)
    \\ &= \mathbb{E}\left[\mathbb{E}\left(\mathbbm{1}\lbrace E \cap Q_n \neq \emptyset \rbrace \prod_{e \in E \cap Q_n} \mathbbm{1} \lbrace e \, \text{is closed} \rbrace \Bigg\vert \Lambda \right)\right]
\end{align*}
The event $\lbrace E \cap Q_n \rbrace$ is $\Lambda$-measurable, hence:
\begin{align}
\label{eq15}
    \mathbb{P}(A \cap B \cap C^c) &\leq \mathbb{E}\left[\mathbbm{1}\lbrace E \cap Q_n \neq \emptyset \rbrace\mathbb{E}\left( \prod_{e \in E \cap Q_n} \mathbbm{1} \lbrace e \, \text{is closed} \rbrace \Bigg\vert \Lambda \right)\right]
\end{align}
Now, we have:
\begin{gather*}
\mathbb{E}\left( \prod_{e \in E \cap Q_n} \mathbbm{1} \lbrace e \, \text{is closed} \rbrace \Bigg\vert \Lambda \right) \\
=\mathbb{E}\left( \prod_{e \in E \cap Q_n} \mathbbm{1} \lbrace \vert e \vert > r \rbrace \mathbbm{1} \lbrace \exists  \, c \subset s, \, \vert c \vert = r \; , \, c  \, \text{topologically closed} \,  ,  X^{\lambda}(c) = 0 \rbrace \Bigg\vert \Lambda \right)
\end{gather*}
For each $e \in E \cap Q_n$, denote:
$$I_e \coloneqq \lbrace \exists \, c \subset s, \; \text{such that} \, \vert c \vert = r \; \text{and} \; c  \; \text{topologically closed} \; \text{and} \,  X^{\lambda}(c) = 0 \rbrace$$
Since the event $\lbrace \vert e \vert > r \rbrace$ is $\Lambda$-measurable for each $e \in E \cap Q_n$, we get:
\begin{equation*}
\mathbb{E}\left( \prod_{e \in E \cap Q_n} \mathbbm{1} \lbrace e \, \text{is closed} \rbrace \Bigg\vert \Lambda \right) = \prod_{e \in E \cap Q_n} \mathbbm{1} \lbrace \vert e \vert > r \rbrace \mathbb{E}\left( \prod_{e \in E \cap Q_n}  \mathbbm{1} \lbrace I_e \rbrace \Bigg\vert \Lambda \right)
\end{equation*}
Conditioned on $\Lambda$, $x^{\lambda}$ has the distribution of a Poisson point process, and the events $\lbrace I_e : e \in E \cap Q_n \rbrace$ depend on the configuration of $X^{\lambda}$ distinct edges. Therefore, by the Poisson independence property, the events $\lbrace I_e : e \in E \cap Q_n \rbrace$ are conditionally independent given $\Lambda$. This yields:
\begin{align}
\nonumber \mathbb{E}\left( \prod_{e \in E \cap Q_n} \mathbbm{1} \lbrace e \, \text{is closed} \rbrace \Bigg\vert \Lambda \right) &= \prod_{e \in E \cap Q_n} \mathbbm{1} \lbrace \vert e \vert > r \rbrace \mathbb{E}\left( \prod_{e \in E \cap Q_n}  \mathbbm{1} \lbrace I_e \rbrace \Bigg\vert \Lambda \right)
\\ \nonumber  &= \prod_{e \in E \cap Q_n} \mathbbm{1} \lbrace \vert e \vert > r \rbrace \mathbb{E}\left(  \mathbbm{1} \lbrace I_e \rbrace \vert \Lambda \right)
\\ \nonumber &= \prod_{e \in E \cap Q_n} \mathbbm{1} \lbrace \vert e \vert > r \rbrace \mathbb{P}\left(   I_e  \vert \Lambda \right)
\\ \label{eq16} &= \prod_{e \in E \cap Q_n} \mathbbm{1} \lbrace \vert e \vert > r \rbrace \prod_{e \in E \cap Q_n}\mathbb{P}\left(   I_e  \vert \Lambda \right)
\end{align}
Since the support will be filled with Cox points as $\lambda \uparrow \infty$, it is clear by monotone convergence that $\displaystyle \forall e \in E \cap Q_n, \, \limsup_{\lambda \uparrow \infty} \mathbb{P}(I_e \vert \Lambda) = 0 \; \text{a.s.}$. Since the cardinal of $E \cap Q_n$ does not depend on $\lambda$ and since the events $\lbrace \vert e \vert > r \; : \; e \in E \cap Q_n \rbrace$ do not depend on $\lambda$ either, using monotone convergence in~\eqref{eq16}, we get:
\begin{align}
   \nonumber  \limsup_{\lambda \uparrow \infty} \mathbb{E}\left( \prod_{e \in E \cap Q_n} \mathbbm{1} \lbrace e \, \text{is closed} \rbrace \Bigg\vert \Lambda \right) &= \left(\prod_{e \in E \cap Q_n} \mathbbm{1} \lbrace \vert e \vert > r \rbrace \right) \mathbbm{1}\lbrace E \cap Q_n = \emptyset \rbrace \; \text{a.s.}
   \\ \label{eq17} &= \mathbbm{1} \lbrace E \cap Q_n = \emptyset \rbrace \; \text{a.s.},
\end{align}
where~\eqref{eq17} is obtained by distinguishing both cases $E \cap Q_n = \emptyset$ and $E \cap Q_n \neq \emptyset$. \\
Getting back to~\eqref{eq15} and applying dominated convergence on the right-hand-side with the use of~\eqref{eq17}, we therefore get:
\begin{equation*}
  \limsup_{\lambda \uparrow \infty} \mathbb{P}(A \cap B \cap C^c) \leq \mathbb{E}\left[\mathbbm{1}\lbrace E \cap Q_n \neq \emptyset \rbrace \mathbbm{1} \lbrace E \cap Q_n = \emptyset \rbrace \right] = 0  
\end{equation*}
Hence:
\begin{equation}
    \label{eq18}
    \limsup_{\lambda \uparrow \infty} \mathbb{P}(A \cap B \cap C^c) = 0
\end{equation}
We now deal with the quantity $\mathbb{P}(A \cap B \cap C \cap F^c)$. We have:
\begin{align*}
    \mathbb{P}(A \cap B \cap C \cap F^c) &\leq \mathbb{P}(F^{c})
    \\ &= \mathbb{E}(\mathbbm{1}\lbrace F^c \rbrace)
    \\ &= \mathbb{E}\left[\mathbb{E}(\mathbbm{1}\lbrace F^c \vert \Lambda ) \right]
    \\ &= \mathbb{E}\left[\mathbb{E}\left( \prod_{v \in V \cap Q_{6n}} \mathbbm{1}\lbrace v \, \text{is open} \rbrace \Bigg\vert \Lambda \right)\right]
\end{align*}
By definition of $Y$, the random variables $(\mathbbm{1} \lbrace v \, \text{is open} \rbrace )_{v \in V}$ are conditionally independent given $\Lambda$, so we get:
\begin{align}
    \nonumber \mathbb{P}(A \cap B \cap C \cap F^c) &\leq \mathbb{E}\left[\prod_{v \in V \cap Q_{6n}}\mathbb{E}\left( \mathbbm{1}\lbrace v \, \text{is open} \rbrace \vert \Lambda \right)\right]
    \\ \nonumber &= \mathbb{E}\left[\prod_{v \in V \cap Q_{6n}}\mathbb{P}\left( v \, \text{is open} \vert \Lambda \right)\right]
    \\ \label{eq19} &= \mathbb{E}\left[(1-p)^{\#(V \cap Q_{6n})}\right], 
\end{align}
where the last line comes directly from the definition of $Y$. By monotone convergence, we have:
\begin{align*}
  \lim_{p \uparrow 1}(1-p)^{\#(V \cap Q_{6n})} &= \mathbbm{1}\lbrace \#(V \cap Q_{6n}) = 0 \rbrace \; \text{a.s.}
  \\ &= \mathbbm{1}\lbrace V \cap Q_{6n} = \emptyset \rbrace \; \text{a.s.}  
\end{align*}
Hence, using dominated convergence in the right hand side of~\eqref{eq19}, we get:
\begin{align*}
\lim_{p \uparrow 1} \mathbb{P}(A \cap B \cap C \cap F^c) &\leq \mathbb{E}(\mathbbm{1}\lbrace V \cap Q_{6n} = \emptyset \rbrace)
\\ &= \mathbb{P}(V \cap Q_{6n} = \emptyset)
\end{align*}
And, finally, by monotone convergence:
\begin{align*}
\lim_{n \uparrow \infty}\lim_{p \uparrow 1} \mathbb{P}(A \cap B \cap C \cap F^c) &\leq \lim_{n \uparrow \infty} \mathbb{P}(V \cap Q_{6n} = \emptyset)
\\ &= \mathbb{P}(V \cap \mathbb{R}^{2} = \emptyset)
\\ &= 0 \quad \text{since $\lambda_{S} > 0$, and so $V \neq \emptyset$}
\end{align*}
Therefore, we have:
\begin{equation}
    \label{eq20}
    \lim_{n \uparrow \infty}\lim_{p \uparrow 1} \mathbb{P}(A \cap B \cap C \cap F^c) = 0
\end{equation}
\bibliographystyle{amsplain}
\bibliography{references}


\end{document} 

